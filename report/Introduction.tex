\section{Введение}
В рамках курса было необходимо разработать приложение, позволяющее 
продемонстрировать применение основных принципов разработки программного 
обеспечения.
В частности, приложение должно содержать следующие компоненты:
\begin{itemize}
	\item слой бизнес-логики;
	\item слой хранения данных;
	\item слой представления.
\end{itemize}
Целью работы являлось разработать систему управления распределенным коллективом, в которой взаимодействуют менеджеры и работники. Работники должны иметь доступ к списку своих задач  с возможностью отмечать их, как выполненные. Менеджеры выдают задания работникам с назначением приоритета, отмечают их, как выполненные, и удаляют. Задачи могут создавать и те, и другие. 

В ходе работы требовалось выполнить следующие задачи:
\begin{enumerate}
	\item Изучить разработку web-приложений с помощью языка Java.
	\item Изучить компоненты системы EJB, необходимые для создания, развертывания и функционирования распределенных приложений Java масштаба предприятия.
	\item Овладеть базовыми навыками разработки распределенных корпоративных Java-приложений в рамках технологии EJB.
	\item Исследовать использование технологии хранения данных (persistence) Java EE 6 для размещения данных приложения в СУБД.
\end{enumerate}
