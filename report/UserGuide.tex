\subsection{Инструкция пользователя}
При запуске приложения пользователь попадает на страницу авторизации, с которой он может перейти на страницу регистрации. В зависимости от типа пользователя, соответствующего паре логин-пароль, главная страница выглядит различно.

\subsubsection{Менеджер}
Главная страница менеджера представлена на рисунке~\ref{fig:manager}. Изначально показаны свободные задачи. Для просмотра задач конкретного работника его требуется выбрать из списка вверху страницы и нажать кнопку <<Choose>>. Для задач указан их приоритет и выполнены ли они (done).

Чтобы назначить работника на свободную задачу, нужно выбрать его из списка рядом с требуемой задачей, ввести целочисленный приоритет и нажать кнопку <<Assign>>. Обратно, для снятия работника с задачи требуется нажать кнопку <<Unassign>> в списке задач этого работника. Свободные задачи можно также удалять нажатием на кнопку <<Delete>>. Чтобы отметить задачу, как выполненную, достаточно нажатия на кнопку <<Finish>>. Удостовериться в успехе можно, снова взглянув на это задание: около него будет написано <<done>>.

\subsubsection{Работник}
Главная страница работника представлена на рисунке~\ref{fig:worker}. Выводится список задач, назначенных этому работнику, с приоритетами и метками о выполнении. Отметить задачу, как выполненную, работник может нажатием кнопки <<Finish>> рядом с задачей, и она будет в списке с пометкой <<done>>.

Оба пользователя могут создавать задачи нажатием кнопки <<addTask>>. Соответствующее окно показано на рисунке \ref{fig:addTask}. Можно ввести описание задачи и добавить её нажатием кнопки <<Submit>>. Система ответит сообщением об успехе. Вернуться обратно можно нажатием на кнопку <<Return>>.