\subsection{Слой хранения данных}
В качестве СУБД для хранения данных использовалась JavaDB. В соответствии с моделью предметной области, были выделены основные сущности: <<пользователь>> и <<задача>>. Были созданы таблицы, соответствующие данным сущностям.

\subsection{Слой бизнес-логики}
Для каждой из сущностей <<пользователь>> и <<задача>> были автоматически сгенерированы классы, соответствующие таблицам, созданным ранее в СУБД. Бизнес-логика расположена в пакете (\textit{dcn.ivanov}), содержащий следующие классы:
\begin{itemize}
	\item \textit{Task.java} - класс, соответствующий сущности <<задача>>
	\item \textit{Worker.java} - класс, соответствующий сущности <<пользователь>>
	\item \textit{WorkplaceSessionBean.java} - реализация интерфейса взаимодействия удаленного пользователя с БД
\end{itemize}