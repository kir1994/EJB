\subsection{Тестирование}
\begin{center}
	\begin{longtable}{|p{0.3\linewidth}|p{0.3\linewidth}|p{0.3\linewidth}|}
		\hline
		\textbf{Вариант тестирования} & \textbf{Ожидаемый результат}&
		\textbf{Фактический результат}\\
		\hline
		\multicolumn{3}{|p{0.3\linewidth}|}{\textit{Менеджер}} \\
		\hline
		Войти с логином и паролем менеджера & Будет открыта главная страница менеджера & Открылась главная страница менеджера  \\
		\hline
		Выбрать работника из списка & Будет показан список задач работника & Показан список задач работника \\
		\hline
	Ввести приоритет, выбрать работника и нажать на кнопку <<Assign>> & Задача появится в списке для работника& 
		Задача появилась в списке для работника\\
		\hline
		Нажать на кнопку <<Finish>> 
		&У задачи появится метка <<done>>& У задачи появилась метка <<done>> \\
		\hline
		Нажать на кнопку <<Delete>>
		&Задача исчезнет из всех списков & Задача исчезла \\
		\hline
		Нажать на кнопку <<Unassign>> 
		&Задача вернется в список свободных & Задача вернулась в список свободных\\
		\hline
		Нажать на кнопку <<AddTask>>, ввести описание задачи и затем <<Return>>
		&Ничего не изменится &Ничего не изменилось\\
		\hline
		Нажать на кнопку <<AddTask>>, ввести описание задачи и затем <<Submit>>
		&Добавится новая задача &Задача добавилась\\
		\hline
		\multicolumn{3}{|p{0.3\linewidth}|}{\textit{Работник}} \\
		\hline
		\hline
		Войти с логином и паролем работника & Будет открыта главная страница работника & Открылась главная страница работника  \\
		\hline
		Нажать на кнопку <<Finished>> у задачи& Появится метка <<done>> &Появилась метка <<done>>\\	
		\hline
	\end{longtable}
\end{center}